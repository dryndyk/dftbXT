\chapter{\dftbp{} API}

You can compile \dftbp{} into a library and access some of its functionality via
an API. Currently the API offers high level access only: you can set the current
geometry and extract energy and forces for that geometry.

\section{Building the library}

In order to compile the \dftbp{} library, issue
\begin{verbatim}
make api
\end{verbatim}
After the compilation, the library (\verb|libdftb+.a|) can be found in the
\verb|api/mm| directory of your build tree. You will find also all mod-files
here. If you issue
\begin{verbatim}
make install_api
\end{verbatim}
the library and the modfiles will be installed in the appropriate
sub-directories under the install target directory. 

You can test the library functionality by issuing
\begin{verbatim}
make test_api
\end{verbatim}


\section{General guidelines}

Although the DFTB+ library contains nearly all internal routines of the DFTB+
code, you should access the code functionality only via the provided API and not
by calling internal routines directly. We aim to keep the API stable over time,
but the internal routines themselves can change without notice between
releases. The API version can be found in the \verb|API_VERSION| file in the
\verb|api/mm| folder. We use semantic versioning, a change in the major (first)
version number indicates backwards incompatible changes, while changes in the
minor (second) version number indicate backwards compatible extensions of the
API.

When using the API, we suggest that \kw{ParserVersion} should be set in order to
ensure that you can maintain backwards compatibility with later versions of
DFTB+.

The Fortran interface is documented in the source code file
\verb|api/mm/mmapi.F90|, while \verb|api/mm/capi.F90| gives the bindings for
calling from C. DFTB+ uses atomic units internally, hence exchanged values
should be in this unit system (however HSD formatted data can carry unit
modifiers, see examples of input parsing for details).
